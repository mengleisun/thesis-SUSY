\documentclass[12pt,twoside]{memoir}
\setcounter{tocdepth}{4}
\setcounter{secnumdepth}{4}
\usepackage[T1]{fontenc}
\usepackage{subfiles}
\usepackage{pennames}
\usepackage{physics}

\usepackage{color}
\usepackage{charter}
\usepackage[expert]{mathdesign}
\usepackage{multirow}
\usepackage{amsmath}
\usepackage{graphicx}
\usepackage{amssymb}
\usepackage{feynmp}
\usepackage{placeins}
\usepackage[numbers,sort]{natbib}

\DeclareGraphicsRule{*}{mps}{*}{}
 
\usepackage[T1]{fontenc}
\usepackage[]{lmodern}

\usepackage[expert]{mathdesign}
\usepackage{stmaryrd}

\usepackage[colorlinks,linkcolor=blue,filecolor=blue,citecolor=blue,urlcolor=blue,backref=page]{hyperref}

\newsubfloat{figure}
\setlength{\parskip}{1em}
% ====================================================================
%     Some formatting suggestions
% ====================================================================
%
\topmargin -0.3in
\oddsidemargin 0.25in
\evensidemargin 0.25in
\textheight 8.5in
\textwidth 6.0in
\linewidth 6.0in
%
% ====================================================================
%     The following commands tent to keep LaTex happier in the 
%     placement of figures, tables, etc
% ====================================================================
%
\renewcommand{\textfraction}{0.0}
\renewcommand{\floatpagefraction}{0.0}
\renewcommand{\topfraction}{1.0}
\renewcommand{\bottomfraction}{1.0}
\setcounter{topnumber}{9}
\setcounter{bottomnumber}{9}
\setcounter{totalnumber}{9}
% 
%%%%%%%%%%%%%% Example for content of file definitions.tex %%%%%%%%%%%
\def\ra    {\rightarrow}
\def\ul    {\underline} 
\def\mevcc {\ifmmode {\mathrm MeV}/c^2 \else MeV$/c^2$\fi}
\def\BsJpsiPhi {\ensuremath{\Bs \ra J/\psi\,\phi}}
%%%%%%%%%%%%%%%%%%%%%%%%%%%%%%%%%%%%%%%%%%%%%%%%%%%%%%%%%%%%%%%%%%%%%%
%
\begin{document} 
%
% ====================================================================
%     Let's first create the title page
% ====================================================================
%
\thispagestyle{empty}
  \begin{center}
    \ \\
    \ \\
    \ \\
    \textbf{
      \LARGE{
        \begin{center}
	Search for supersymmetry using events with a photon, a
	lepton, and missing transverse momentum in proton-proton collisions at \\ $\sqrt{s} =$ 13 TeV in CMS
        \end{center}
      }
    }
    \ \\
    \ \\
    \ \\
    by \\
    \ \\
    \large{Menglei Sun} \\
    \ \\
    \ \\
    \ \\
    \ \\
    \large{
      Submitted in partial fulfillment of the \\
      requirements for the degree of \\
      Doctor of Philosophy \\
      at \\
      Carnegie Mellon University \\
      Department of Physics \\
      Pittsburgh, Pennsylvania \\
    }
    \ \\
    \ \\
    Advised by Professor Manfred Paulini \\
    \ \\
    \today
  \end{center}

% ====================================================================
%     Generates one blank page before the 'Abstract'
% ====================================================================

\thispagestyle{empty} \cleardoublepage

% ====================================================================
%     Input your abstract in file abstract.tex here. 
%     A simple example of the content of such a file is given below.
% ====================================================================
%
%\include{abstract}
%
%%%%%%%%%%%%%% Example for content of file abstract.tex %%%%%%%%%%%%%%
\begin{abstract}
    Results of a search for new physics in events with a
    photon, an electron or muon, and large missing transverse energy (\MET) is presented.
    The study is based on a sample of proton-proton collisions at $\sqrt{s} =  
     13$ TeV corresponding to an integrated luminosity of 35.9 $fb^{-1}$ collected
    with the CMS detector in 2016. Many models of new physics predict events
    with significant \MET~in addition to electroweak gauge bosons.
    Models of supersymmetry (SUSY) with gauge-mediated supersymmetry
    breaking (GMSB) naturally yield events with photons in the final state.
    Searches for events with both a photon and a lepton enhanced the sensitivity 
    to electroweak production of supersymmetric particles. We interpret the results of our search
    in the context of SUSY with gauge-mediated supersymmetry breaking as well as simplified SUSY models.
\end{abstract}
%%%%%%%%%%%%%%%%%%%%%%%%%%%%%%%%%%%%%%%%%%%%%%%%%%%%%%%%%%%%%%%%%%%%%%

\thispagestyle{empty} \cleardoublepage
\pagenumbering{roman}

% ====================================================================
%     Input your acknowledgments in file acknowledgments.tex here. 
%     A simple example of the content of such a file is given below.
% ====================================================================
%
%\include{acknowledgments}
%
\section*{Acknowledgements}
First of all I would like to express my sincere gratitude to my advisor, Manfred Paulini, for guiding and inspiring me during my graduate studies. Manfred has always been very supportive of students trying new ideas. He also encourages us to attend conferences and collaborate with other researchers. Thanks to Manfred, I have got the opportunity to participate in many interesting research topics, including the study of the Phase II upgrade of the CMS detector and the search for supersymmetry. 

I would like to thank the conveners of the SUSY photon subgroup: Andrew Askew, Si Xie, Rishi Patel, and Marc Weinberg. Their comments and suggestions to this thesis research have been very helpful in pushing the analysis forward. I would like to thank particularly Marc, who is also a co-author of this analysis. I benefit a lot from his wide knowledge on the supersymmetry search and CMS detector. It's a pleasure to work with him.  

Working in a large group like the ECAL subdetector group is both enjoyable and challenging. I would like to express my appreciation to the leaders of the ECAL group, Dave Barney and David Petyt, for their helping and supporting on ECAL related work. My gratitude also goes to other members of the Carnegie Mellon group, Tanmay Mudholkar and Michael Andrews, whose dedicated work on the ECAL and DQM systems are crucial for the successful data taking of the CMS experiment.  

I would also like to thank Yutaro Iiyama, who has helped me a lot on academic work. When I first came to CERN in 2014, I was just a beginner of the CMS, feeling nervous about presenting my work in meetings and taking detector-on-call shifts. Yutaro taught me a lot of useful things about the ECAL detector and the DQM system. With his help, I gradually became the on-call expert of the ECAL and DQM system. He is also very responsive when I have questions about the physics analysis. 

Doing research in CERN was an exciting experience, and the life in Geneva became even more wonderful when I met a lot of fabulous friends there. 
To my dear friends Mingjian Lu and Peng Jiang, who was also my housemates, I would like to thank them for the discussions we ever had and all the joy they brought to my life. 
It was a tragedy that we lost them in an accident. I will always miss them. 
Luckily, I still have the support and accompany from other friends. 
Here I would like to thank my friends, Hua Wei, Xinmei Niu, Haonan Lu, Qi Zend, and Liangjing Zhu, for their support and help. 
I also benefit a lot from the discussion of physics problems with these friends. 

My deepest gratitude to all my family, particularly to my beloved husband, Jingkun Gao, for his love and encouragement; and to my parents, for their support and understanding.  


\newpage
\tableofcontents
\listoftables
\listoffigures

\clearpage 

\pagenumbering{arabic}

\chapter{Introduction}
What is the world made of? This is a question that has been asked by humans for thousands of years. 
The idea that all matter is composed of elementary particles dates from at least the 5th century BC. 
The development of the atomic theory at the start of the 19th century showed that all materials are 
made of atoms, which were thought to be the fundamental particles. However, the discovery of the electron 
in 1896, the nucleus in 1911 and the neutron in 1931 revealed that the atoms are made of smaller sub-atomic particles. 
To better understand the structure of matter, physicists need to look deeper and study the most elementary particles. 
Particle physics is a branch of the physics that aims to understand the fundamental nature of matter and energy, 
the interactions between them, and to apply that knowledge to better understand the origin and structure of the universe. 
Throughout the 1950s and 1960s, hundreds of sub-atomic particles were discovered in particle physics experiments.
In examing and organizing these particles, a Standard Model of the particle physics has emerged.

The Standard Model is our current theory that best describes the subatomic world. 
According to the Standard Model, matter is built up from a set of spin-1/2 particles, called fermions. 
Forces between matter result from the exchange of spin-1 particles, called bosons. 
In addition, the Standard Model predicts a spin-0 particle, called the Higgs boson, which explains the origin of particle mass. 
Since its formulation in the 1970s, the Standard Model has stood up many experimental tests. 
The Higgs boson, which is the final piece of the Standard Model, has been discovered in 2012 by the ATLAS and CMS experiments. 
This discovery marks the triumph of the Standard Model. 

Despite its great success, the Standard Model still leaves open many unanswered questions. 
What is the cause of the significant asymmetry between matter and antimatter? 
Why is the Higgs mass so small while it is sensitive to the ultraviolet physics through radiative corrections?
Moreover, a variety of astronomical and cosmological experiments provide strong evidence of the existence of dark matter and dark energy. 
It turns out that the ``normal'' matter only accounts for about 5\% of the universe. 
So what is the nature of the dark matter and dark energy that fill up the rest of the universe? 
To solve these problems, many new physics models are explored, each extending the Standard Model using a different mechanism. 
The supersymmetric extension of the Standard Model is an appealing theory that offers solutions to these problems.
It proposes a symmetry between fermions and bosons and doubles the number of particles in Standard Model. 
The new set of particles predicted by the Symmetry Model help stabilize the mass of Higgs boson, and the lightest Supersymmetry particle provides a candidate for dark matter. 

The Large Hadron Collider (LHC) at CERN was built to search for Higgs boson and new physics beyond the Standard Model.
In 2015, the LHC started its second run with an increased center-of-mass energy of 13 TeV, providing proton-proton collisions every 25 ns. 
This powerful machine provides us the opportunities to probe physics at TeV scale. 
It hosts seven detectors, each designed for certain kinds of researches. 
The Compact Muon Solenoid (CMS) is a general-purpose detector sitting in one of the LHC collision points. 
The essential design of the CMS is the use of a 3.8 T superconducting magnet that can provide a large bending power and enable a precise momentum measurement. 
This thesis presents an analysis using the proton-proton collision data collected by the CMS in 2016. 

The analysis presented in this thesis searches for supersymmetry using events with one photon, one lepton, and large transverse missing momentum.  
Events with photons are typical signatures of supersymmetry scenarios with gauge-mediated SUSY breaking.
Final states with an additional lepton suppress the Standard Model backgrounds and enhance the sensitivity to the electroweak production of supersymmetric particle.
The remaining backgrounds from standard model processes are estimated using both simulation and data-driven methods. 

This thesis is organized as follows: Chapter \ref{ch1} gives a brief introduction to the Standard Model and its supersymmetric extension. Chapter \ref{ch2} describes the LHC and the CMS detector, while the reconstruction of data is given in Chapter \ref{ch3}. Chapter \ref{ch4} describes the data used in this analysis, and chapter  \ref{ch5} presents the selection of the data. Background estimations are discussed in detailed in chapter \ref{ch6}. The results of the search and its interpretations are presented in Chapter \ref{ch7}. Finally, the thesis is summarized in Chapter \ref{ch8}.

 
\subfile{StandardModel}
\subfile{Detector}
\subfile{Reconstruction}
\subfile{Datasets}
\subfile{EventSelection}
\subfile{Backgrounds}
\subfile{Results}
\subfile{Conclusions}

\newpage
\bibliographystyle{auto_generated}
\bibliography{reference}

\end{document}
