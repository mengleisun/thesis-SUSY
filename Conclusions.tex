\documentclass[thesis.tex]{subfiles}
\renewcommand\_{\textunderscore\allowbreak}

\begin{document}


\chapter{Conclusions}
\label{ch8}

A search for SUSY with general gauge mediation in events with at least one lepton, one photon and large missing transverse momentum using proton-proton collisions at $\sqrt{s} = $ 13 TeV has been presented. 
The data sample, collected in 2016 with the CMS detector at the CERN LHC, corresponds to an integrated luminosity of 35.9 $fb^{-1}$. 

The search selects events online using diphoton and photon+muon triggers, and applies an offline selection using kinematic requirements on photon, lepton and missing transverse momentum.  
Signal candidate events in the $\MET > $ 120 GeV and $M_T >$ 100 GeV search region are counted in multiple bins of $p_T^\gamma$, $H_T$, and \MET. 
The estimation of the SM backgrounds are performed using data-driven methods and MC simulations, while the estimation methods are verified in a dedicated validation region. 

No significant excess above the SM background is observed in the signal region.
The results are interpreted in GGM models as well as simplified models motivated by gauge-mediated supersymmetry breaking. 
GGM scans are performed in an M1 and M2 parameter space where model points up to M2 = 1200 GeV are excluded.
For strong production simplified models, gluino production up to 1700 GeV and squarks up to 1400 GeV can be excluded at 95\% confidence level. 
Final states with an additional lepton enhance the sensitivity to electroweak production of SUSY particles. 
For the TChiWg model, where $\PSGczDo\PSGcpmDo$ are pair produced, a NLSP of mass of up to 900 GeV is excluded, extending the current best limit by about 120 GeV. 

The future of this search appears very promising. 
With more data, the sensitivity to the SUSY signals can be enhanced and the estimation of the SM backgrounds can be improved.  
A possible future improvement of this search is to use data-driven method to estimate the contributions from $t\bar{t}\gamma$, $WW\gamma$, and $WZ\gamma$ in a control region formed by one photon and two leptons. 
Such methods would greatly benefit from the increasing amount of data. 
Up to today, the LHC has delivered a total of 136.9~$fb^{-1}$ of proton-proton collisions at $\sqrt{s} = $ 13 TeV. 
The Phase I upgrade of the LHC is scheduled after the data-taking in 2018, and the LHC will restart with ultimate design luminosity after this upgrade.  
A full exploration of these data will bring us closer to new physics.

\end{document}
